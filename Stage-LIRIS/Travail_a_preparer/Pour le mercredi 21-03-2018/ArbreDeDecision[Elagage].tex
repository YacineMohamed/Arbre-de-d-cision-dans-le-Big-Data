\documentclass[a4paper, 11pt]{report}
\setcounter{tocdepth}{3}
\usepackage[utf8]{inputenc}
\usepackage[french]{babel}
\usepackage[T1]{fontenc}
\usepackage{graphicx}
\usepackage{xcolor}
\usepackage{booktabs}
\usepackage{tabularx}
\usepackage{fourier} 
\usepackage{array}
\usepackage{makecell}
\usepackage[top=2.5cm,bottom=2.5cm,right=2.5cm,left=2.5cm]{geometry}
\usepackage{amsmath}
\usepackage{xcolor}
\usepackage{colortbl,hhline}
\usepackage{subfig}
\usepackage{multirow}
\usepackage{comment}
\usepackage{makecell}

\usepackage{algorithmic,algorithm}

\renewcommand{\listalgorithmname}{Liste des \ALG@name s}

\definecolor{yacine}{RGB}{241,241,241}
\usepackage{tikz}
\def\checkmark{\tikz\fill[scale=0.4](0,.35) -- (.25,0) -- (1,.7) -- (.25,.15) -- cycle;} 

\usepackage[backend=bibtex,style=numeric,sorting=nty]{biblatex}
\nocite{*} %Ausgabe aller Bibliographieeinträges
\usepackage{hyperref}
\hypersetup{
    linktoc=all,     %set to all if you want both sections and subsections linked
}




\begin{document}
 
\begin{itemize}
\item Problématique : \\
Après sa création, l'un arbre de décision est généralement exposé à deux problèmes, à savoir : sa complexité (en terme de taille), ainsi que le problème du sur-apprentissage. Faisans partie des algorithmes d'apprentissage supervisé, les arbres dépendent naturellement d'un ensemble d'apprentissage afin de construire leur modèle prédictif. 
\\Le sur-apprentissage se caractérise par une très bonne performance de l'arbre  sur cet ensemble de données d'apprentissage, ainsi qu'une mauvaise sur d'autre ensemble (ensemble de teste).\\
Afin de palier a ce problème, l'arbre de décision doit être élagué, cet élagage peut se faire de deux manières différentes : \\
\begin{enumerate}
\item Pre-Elagage \\
Comme son nom l'indique, cette technique permet d'élaguer l'arbre pendant sa création, et ce, en ajoutant des conditions d'arrêts, ce qui permettra l'arrêt du développement d'un sous ensemble de données avant d'atteindre l'homogénéité totale, si la condition d'arrêt est vérifiée a son niveau.\\
la condition pourrait être (entre autres) :
\begin{itemize}
\item Un seuil minimal d'instance\\
Qui, une fois atteint, le nœud est considéré comme étant une feuille.
\item Longueur maximale de l'arbre\\
Ou seuls les nœuds ne dépassant pas cette longueur maximal seront développés.
\end{itemize}
\item Post-Elagage\\
Concernant cette technique, l'élagage de l'arbre se fait qu'après sa création,, et ce, en supprimant des sous arbre complet, en les remplaçant par les feuilles qui représentes la classe la plus fréquente  au niveau de ce sous arbre.\\
Le parcours se fait de la racine a la feuille, pour chaque nœud (feuille ou interne), la précision de l'arbre est évaluée (calculée) avant et après sa suppression, si la différence est grande, le sous arbre sera élagué et remplacé donc par une feuille. 
\\Éventuellement, d'autres nœuds seront perdus lors de cette suppression, cela ne va pas affecté la précision, qui a la base devait être augmentée?\\
Avec le calcule et la comparaison de la précision de l'arbre avant et après la suppression du sous arbre,
Si la précision de l'arbre était meilleur (égale ou légèrement moins bonne) avec ces nœuds, le sous arbre n'aurait pas été supprimé, nous somme donc sûr que la suppression du sous arbre ne ferai qu'améliorer la précision de l'arbre
\end{enumerate}
\end{itemize}

\end{document}
