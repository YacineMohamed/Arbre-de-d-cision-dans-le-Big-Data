\documentclass[a4paper, 11pt]{report}
\setcounter{tocdepth}{3}
\usepackage[utf8]{inputenc}
\usepackage[french]{babel}
\usepackage[T1]{fontenc}
\usepackage{graphicx}
\usepackage{xcolor}
\usepackage{booktabs}
\usepackage{tabularx}
\usepackage{fourier} 
\usepackage{array}
\usepackage{makecell}
\usepackage[top=2.5cm,bottom=2.5cm,right=2.5cm,left=2.5cm]{geometry}
\usepackage{amsmath}
\usepackage{xcolor}
\usepackage{colortbl,hhline}
\usepackage{subfig}
\usepackage{multirow}
\usepackage{comment}
\usepackage{makecell}

\usepackage{algorithmic,algorithm}

\renewcommand{\listalgorithmname}{Liste des \ALG@name s}

\definecolor{yacine}{RGB}{241,241,241}
\usepackage{tikz}
\def\checkmark{\tikz\fill[scale=0.4](0,.35) -- (.25,0) -- (1,.7) -- (.25,.15) -- cycle;} 

\usepackage[backend=bibtex,style=numeric,sorting=nty]{biblatex}
\nocite{*} %Ausgabe aller Bibliographieeinträges
\usepackage{hyperref}
\hypersetup{
    linktoc=all,     %set to all if you want both sections and subsections linked
}




\begin{document}
\section{Gestion d'attributs continus} 


Soit l'ensemble de données suivant :
\begin{center}
\begin{tabular}{| c | c | c | c |}
\hline
Température & Faiblesse & Toux & Malade ?\\
\hline
10 & 0 & oui & Angines\\
\hline
30 & 10 & non & Non\_malade\\
\hline
20 & 10 & oui & Angines\\
\hline

30 & 10 & non & Non\_malade\\
\hline

40 & 0 & non & Fievre\\
\hline

20 & 10 & oui & Angines\\
\hline
\end{tabular}
\end{center}
L'algorithme C4.5 gère les attribut continus de la manière suivante : 
\begin{itemize}
\item Étape 01 : Trier les valeur de l'attribut\\
Ce qui donnera le résultat suivant : 
\begin{tabular}{| c | c | c | c |}
\hline
10 & 20 & 30 & 40\\
\hline
\end{tabular}

\item Étape 02 : Créer un seuil a partir de chaque valeurs consécutives, ce qui donnera le résultat suivant :\\
$\frac{10+20}{2} = 15, \ \frac{20+30}{2} = 25, \ \frac{30+40}{2} = 35$

\item Etape 03 : Calculer la quantité d'information apporté par chacune des valeurs précédente, en transformant les valeurs initiales de l'attribut en valeurs $(\ge VAL)$ ou $(< VAL)$
\begin{enumerate}
\item Valeur : 15


\begin{center}
\begin{tabular}{| c | c | c | c |}
\hline
Température & Faiblesse & Toux & Malade ?\\
\hline
$< 15$ & 0 & oui & Angines\\
\hline
$\ge 15$ & 10 & non & Non\_malade\\
\hline
$\ge 15$ & 10 & oui & Angines\\
\hline

$\ge 15$ & 10 & non & Non\_malade\\
\hline

$\ge 15$ & 0 & non & Fievre\\
\hline

$\ge 15$ & 10 & oui & Angines\\
\hline
\end{tabular}
\end{center}

Entropie(Température) = $\frac{1}{6}( - \frac{1}{1}Log_2(\frac{1}{1}))+\frac{5}{6}(- (\frac{1}{5}
Log_2(\frac{1}{5})+ \frac{2}{5}Log_2(\frac{2}{5})+\frac{2}{5}Log_2(\frac{2}{5}))$\\
Entropie(Température) = 1.27

\item Valeur : 25 

\begin{center}
\begin{tabular}{| c | c | c | c |}
\hline
Température & Faiblesse & Toux & Malade ?\\
\hline
$<25$ & 0 & oui & Angines\\
\hline
$\ge 25$ & 10 & non & Non\_malade\\
\hline
$<25$ & 10 & oui & Angines\\
\hline

$\ge 25$ & 10 & non & Non\_malade\\
\hline

$\ge 25$ & 0 & non & Fievre\\
\hline

$<25$ & 10 & oui & Angines\\
\hline
\end{tabular}
\end{center}

Entropie(Température) = $\frac{3}{6}( - \frac{3}{3}Log_2(\frac{3}{3}))+\frac{3}{6}(- (\frac{1}{3}
Log_2(\frac{1}{3})+ \frac{2}{3}Log_2(\frac{2}{3}))$\\
Entropie(Température) = 0.48




\item Valeur : 35


\begin{center}
\begin{tabular}{| c | c | c | c |}
\hline
Température & Faiblesse & Toux & Malade ?\\
\hline
$<35$ & 0 & oui & Angines\\
\hline
$<35$ & 10 & non & Non\_malade\\
\hline
$<35$ & 10 & oui & Angines\\
\hline

$<35$ & 10 & non & Non\_malade\\
\hline

$\ge 30$ & 0 & non & Fievre\\
\hline

$<35$ & 10 & oui & Angines\\
\hline
\end{tabular}
\end{center}

Entropie(Température) = $\frac{1}{6}( - \frac{1}{1}Log_2(\frac{1}{1}))+\frac{5}{6}(- (\frac{2}{5}Log_2(\frac{2}{5})+ \frac{3}{5}Log_2(\frac{3}{5}))$\\
Entropie(Température) = 0.81
\end{enumerate}
Le seuil choisi sera donc $\ge 25, \ <25$, l'ensemble d'apprentissage qui sera utilisé sera donc le suivant: \\

\begin{center}
\begin{tabular}{| c | c | c | c |}
\hline
Température & Faiblesse & Toux & Malade ?\\
\hline
$<25$ & 0 & oui & Angines\\
\hline
$\ge 25$ & 10 & non & Non\_malade\\
\hline
$<25$ & 10 & oui & Angines\\
\hline

$\ge 25$ & 10 & non & Non\_malade\\
\hline

$\ge 25$ & 0 & non & Fievre\\
\hline

$<25$ & 10 & oui & Angines\\
\hline
\end{tabular}
\end{center}



\end{itemize}


\end{document}
