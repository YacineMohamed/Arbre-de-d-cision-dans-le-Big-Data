\documentclass[a4paper, 11pt]{report}
\setcounter{tocdepth}{3}
\usepackage[utf8]{inputenc}
\usepackage[french]{babel}
\usepackage[T1]{fontenc}
\usepackage{graphicx}
\usepackage{xcolor}
\usepackage{booktabs}
\usepackage{tabularx}
\usepackage{fourier} 
\usepackage{array}
\usepackage{makecell}
\usepackage[top=2.5cm,bottom=2.5cm,right=2.5cm,left=2.5cm]{geometry}
\usepackage{amsmath}
\usepackage{xcolor}
\usepackage{colortbl,hhline}
\usepackage{subfig}
\usepackage{multirow}
\usepackage{comment}
\usepackage{makecell}

\usepackage{algorithmic,algorithm}

\renewcommand{\listalgorithmname}{Liste des \ALG@name s}

\definecolor{yacine}{RGB}{241,241,241}
\usepackage{tikz}
\def\checkmark{\tikz\fill[scale=0.4](0,.35) -- (.25,0) -- (1,.7) -- (.25,.15) -- cycle;} 

\usepackage[backend=bibtex,style=numeric,sorting=nty]{biblatex}
\nocite{*} %Ausgabe aller Bibliographieeinträges
\usepackage{hyperref}
\hypersetup{
    linktoc=all,     %set to all if you want both sections and subsections linked
}




\begin{document}
 
\begin{center}
\begin{tabular}{|c|}
\hline
Compte rendu de la séance \\du 23\//03\//2018\\
\hline

\end{tabular}
\end{center}


En présence de :\\
\begin{itemize}
\item Monsieur Emmanuel COQUERY
\end{itemize}
Durant cette séance, Mr COQUERY m'a tout d'abord expliqué (avec des exemple sur machine) l'importance de bien structurer les notes dans un seul fichier, en utilisant un outils puissant dédié à cela (Emacs, Visuel Studio Code). Ce qui permettra d'avoir un accès très rapide à n'importe quelle information.
\begin{itemize}
\item Choisir un outils et commencer à l'utiliser avant le prochain rendez-vous.
\end{itemize}
En ce qui concerne le travail que je devais préparer, Mr Coquery m'a demandé de revoir les points suivant:
\begin{enumerate}
\item Deuxième paragraphe de la réponse N°1
\begin{table}[h!]
\begin{center}
\begin{tabular}{|l|}
\hline 
\rowcolor{gray!25}
Ce qui rend impossible la définition d'un ensemble parfait, vu le nombre infini\\
\rowcolor{gray!25}
de possibilité que l'on peut avoir, en variant les paramètres (nombre de lignes,\\
\rowcolor{gray!25}
nombre d'attribut, nombre de catégorie de classes, ... etc)\\
\hline
\end{tabular}
\end{center}
\end{table}

\item Faire une recherche plus raffinée concernant les caractéristiques des ensembles de données, tout en définissant chacune (de manière formel).
\item Ne plus utiliser "...etc".
\item S'appuyer sur les formules mathématique afin de renforcer les définitions.
\item Structuré tout ce qui a déjà été fait dans un seul document.

\end{enumerate}

Travail à préparer pour le lundi 26/03/2018, à 15:00

\end{document}
