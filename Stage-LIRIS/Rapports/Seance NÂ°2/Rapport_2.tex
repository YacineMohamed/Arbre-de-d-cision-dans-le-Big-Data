\documentclass[a4paper, 11pt]{report}
\setcounter{tocdepth}{3}
\usepackage[utf8]{inputenc}
\usepackage[french]{babel}
\usepackage[T1]{fontenc}
\usepackage{graphicx}
\usepackage{xcolor}
\usepackage{booktabs}
\usepackage{tabularx}
\usepackage{fourier} 
\usepackage{array}
\usepackage{makecell}
\usepackage[top=2.5cm,bottom=2.5cm,right=2.5cm,left=2.5cm]{geometry}
\usepackage{amsmath}
\usepackage{xcolor}
\usepackage{colortbl,hhline}
\usepackage{subfig}
\usepackage{multirow}
\usepackage{comment}
\usepackage{makecell}

\usepackage{algorithmic,algorithm}

\renewcommand{\listalgorithmname}{Liste des \ALG@name s}

\definecolor{yacine}{RGB}{241,241,241}
\usepackage{tikz}
\def\checkmark{\tikz\fill[scale=0.4](0,.35) -- (.25,0) -- (1,.7) -- (.25,.15) -- cycle;} 

\usepackage[backend=bibtex,style=numeric,sorting=nty]{biblatex}
\nocite{*} %Ausgabe aller Bibliographieeinträges
\usepackage{hyperref}
\hypersetup{
    linktoc=all,     %set to all if you want both sections and subsections linked
}




\begin{document}
 
\begin{center}
\begin{tabular}{|c|}
\hline
Compte rendu de la séance \\du 21\//03\//2018\\
\hline

\end{tabular}
\end{center}


En présence de :\\
\begin{itemize}
\item Monsieur Emmanuel COQUERY

Durant cette séance, Mr COQUERY a tout d'abord abordé les points qui restait flous lors de la séance précédentes, à savoir : 
\begin{itemize}
\item Les métriques qui permettent d'évaluer la qualité d'un arbre de décision\\
A ce sujet, ma réponse était de présenter quelques métriques (Précision, vitesse, taille) en précisant que l'évaluation restera toujours relative au contexte.\\
Point à détailler d'avantage
\item L'impact de l'élagage sur la précision global d'un arbre de décision\\
Nous somme arrivé a la conclusion qu'une condition doit être vérifiée avant d'élaguer un sous arbre, la condition consiste à évaluer l'impacte de l'élagage, en comparant la précision de l'arbre, avant et après élagage.
\item Les différents critères de division
Trois points qui sont détaillés dans les documents du dossier correspondant a cette séance.
\end{itemize}

\item A préparer pour le vendredi 23/03/2018:\\
\begin{enumerate}
\item Faire une recherche sur les caractéristiques qu'un ensemble de données peut avoir, ainsi que les critères qui peuvent être utilisés afin d'évaluer les données d'apprentissage ( Taille des données, nombre de ligne, nombre d'attribut, nombre de catégorie de classes, ...)

\item Le nettoyage de données est-il nécessaire avant la construction du modèle ? est-il utilisé par un des algorithmes existant? lequel ? 

\end{enumerate}

\end{itemize}

\end{document}
